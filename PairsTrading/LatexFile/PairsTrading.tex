\documentclass[11pt]{article}

    \usepackage[breakable]{tcolorbox}
    \usepackage{parskip} % Stop auto-indenting (to mimic markdown behaviour)
    
    \usepackage{iftex}
    \ifPDFTeX
    	\usepackage[T1]{fontenc}
    	\usepackage{mathpazo}
    \else
    	\usepackage{fontspec}
    \fi

    % Basic figure setup, for now with no caption control since it's done
    % automatically by Pandoc (which extracts ![](path) syntax from Markdown).
    \usepackage{graphicx}
    % Maintain compatibility with old templates. Remove in nbconvert 6.0
    \let\Oldincludegraphics\includegraphics
    % Ensure that by default, figures have no caption (until we provide a
    % proper Figure object with a Caption API and a way to capture that
    % in the conversion process - todo).
    \usepackage{caption}
    \DeclareCaptionFormat{nocaption}{}
    \captionsetup{format=nocaption,aboveskip=0pt,belowskip=0pt}

    \usepackage[Export]{adjustbox} % Used to constrain images to a maximum size
    \adjustboxset{max size={0.9\linewidth}{0.9\paperheight}}
    \usepackage{float}
    \floatplacement{figure}{H} % forces figures to be placed at the correct location
    \usepackage{xcolor} % Allow colors to be defined
    \usepackage{enumerate} % Needed for markdown enumerations to work
    \usepackage{geometry} % Used to adjust the document margins
    \usepackage{amsmath} % Equations
    \usepackage{amssymb} % Equations
    \usepackage{textcomp} % defines textquotesingle
    % Hack from http://tex.stackexchange.com/a/47451/13684:
    \AtBeginDocument{%
        \def\PYZsq{\textquotesingle}% Upright quotes in Pygmentized code
    }
    \usepackage{upquote} % Upright quotes for verbatim code
    \usepackage{eurosym} % defines \euro
    \usepackage[mathletters]{ucs} % Extended unicode (utf-8) support
    \usepackage{fancyvrb} % verbatim replacement that allows latex
    \usepackage{grffile} % extends the file name processing of package graphics 
                         % to support a larger range
    \makeatletter % fix for grffile with XeLaTeX
    \def\Gread@@xetex#1{%
      \IfFileExists{"\Gin@base".bb}%
      {\Gread@eps{\Gin@base.bb}}%
      {\Gread@@xetex@aux#1}%
    }
    \makeatother

    % The hyperref package gives us a pdf with properly built
    % internal navigation ('pdf bookmarks' for the table of contents,
    % internal cross-reference links, web links for URLs, etc.)
    \usepackage{hyperref}
    % The default LaTeX title has an obnoxious amount of whitespace. By default,
    % titling removes some of it. It also provides customization options.
    \usepackage{titling}
    \usepackage{longtable} % longtable support required by pandoc >1.10
    \usepackage{booktabs}  % table support for pandoc > 1.12.2
    \usepackage[inline]{enumitem} % IRkernel/repr support (it uses the enumerate* environment)
    \usepackage[normalem]{ulem} % ulem is needed to support strikethroughs (\sout)
                                % normalem makes italics be italics, not underlines
    \usepackage{mathrsfs}
    

    
    % Colors for the hyperref package
    \definecolor{urlcolor}{rgb}{0,.145,.698}
    \definecolor{linkcolor}{rgb}{.71,0.21,0.01}
    \definecolor{citecolor}{rgb}{.12,.54,.11}

    % ANSI colors
    \definecolor{ansi-black}{HTML}{3E424D}
    \definecolor{ansi-black-intense}{HTML}{282C36}
    \definecolor{ansi-red}{HTML}{E75C58}
    \definecolor{ansi-red-intense}{HTML}{B22B31}
    \definecolor{ansi-green}{HTML}{00A250}
    \definecolor{ansi-green-intense}{HTML}{007427}
    \definecolor{ansi-yellow}{HTML}{DDB62B}
    \definecolor{ansi-yellow-intense}{HTML}{B27D12}
    \definecolor{ansi-blue}{HTML}{208FFB}
    \definecolor{ansi-blue-intense}{HTML}{0065CA}
    \definecolor{ansi-magenta}{HTML}{D160C4}
    \definecolor{ansi-magenta-intense}{HTML}{A03196}
    \definecolor{ansi-cyan}{HTML}{60C6C8}
    \definecolor{ansi-cyan-intense}{HTML}{258F8F}
    \definecolor{ansi-white}{HTML}{C5C1B4}
    \definecolor{ansi-white-intense}{HTML}{A1A6B2}
    \definecolor{ansi-default-inverse-fg}{HTML}{FFFFFF}
    \definecolor{ansi-default-inverse-bg}{HTML}{000000}

    % commands and environments needed by pandoc snippets
    % extracted from the output of `pandoc -s`
    \providecommand{\tightlist}{%
      \setlength{\itemsep}{0pt}\setlength{\parskip}{0pt}}
    \DefineVerbatimEnvironment{Highlighting}{Verbatim}{commandchars=\\\{\}}
    % Add ',fontsize=\small' for more characters per line
    \newenvironment{Shaded}{}{}
    \newcommand{\KeywordTok}[1]{\textcolor[rgb]{0.00,0.44,0.13}{\textbf{{#1}}}}
    \newcommand{\DataTypeTok}[1]{\textcolor[rgb]{0.56,0.13,0.00}{{#1}}}
    \newcommand{\DecValTok}[1]{\textcolor[rgb]{0.25,0.63,0.44}{{#1}}}
    \newcommand{\BaseNTok}[1]{\textcolor[rgb]{0.25,0.63,0.44}{{#1}}}
    \newcommand{\FloatTok}[1]{\textcolor[rgb]{0.25,0.63,0.44}{{#1}}}
    \newcommand{\CharTok}[1]{\textcolor[rgb]{0.25,0.44,0.63}{{#1}}}
    \newcommand{\StringTok}[1]{\textcolor[rgb]{0.25,0.44,0.63}{{#1}}}
    \newcommand{\CommentTok}[1]{\textcolor[rgb]{0.38,0.63,0.69}{\textit{{#1}}}}
    \newcommand{\OtherTok}[1]{\textcolor[rgb]{0.00,0.44,0.13}{{#1}}}
    \newcommand{\AlertTok}[1]{\textcolor[rgb]{1.00,0.00,0.00}{\textbf{{#1}}}}
    \newcommand{\FunctionTok}[1]{\textcolor[rgb]{0.02,0.16,0.49}{{#1}}}
    \newcommand{\RegionMarkerTok}[1]{{#1}}
    \newcommand{\ErrorTok}[1]{\textcolor[rgb]{1.00,0.00,0.00}{\textbf{{#1}}}}
    \newcommand{\NormalTok}[1]{{#1}}
    
    % Additional commands for more recent versions of Pandoc
    \newcommand{\ConstantTok}[1]{\textcolor[rgb]{0.53,0.00,0.00}{{#1}}}
    \newcommand{\SpecialCharTok}[1]{\textcolor[rgb]{0.25,0.44,0.63}{{#1}}}
    \newcommand{\VerbatimStringTok}[1]{\textcolor[rgb]{0.25,0.44,0.63}{{#1}}}
    \newcommand{\SpecialStringTok}[1]{\textcolor[rgb]{0.73,0.40,0.53}{{#1}}}
    \newcommand{\ImportTok}[1]{{#1}}
    \newcommand{\DocumentationTok}[1]{\textcolor[rgb]{0.73,0.13,0.13}{\textit{{#1}}}}
    \newcommand{\AnnotationTok}[1]{\textcolor[rgb]{0.38,0.63,0.69}{\textbf{\textit{{#1}}}}}
    \newcommand{\CommentVarTok}[1]{\textcolor[rgb]{0.38,0.63,0.69}{\textbf{\textit{{#1}}}}}
    \newcommand{\VariableTok}[1]{\textcolor[rgb]{0.10,0.09,0.49}{{#1}}}
    \newcommand{\ControlFlowTok}[1]{\textcolor[rgb]{0.00,0.44,0.13}{\textbf{{#1}}}}
    \newcommand{\OperatorTok}[1]{\textcolor[rgb]{0.40,0.40,0.40}{{#1}}}
    \newcommand{\BuiltInTok}[1]{{#1}}
    \newcommand{\ExtensionTok}[1]{{#1}}
    \newcommand{\PreprocessorTok}[1]{\textcolor[rgb]{0.74,0.48,0.00}{{#1}}}
    \newcommand{\AttributeTok}[1]{\textcolor[rgb]{0.49,0.56,0.16}{{#1}}}
    \newcommand{\InformationTok}[1]{\textcolor[rgb]{0.38,0.63,0.69}{\textbf{\textit{{#1}}}}}
    \newcommand{\WarningTok}[1]{\textcolor[rgb]{0.38,0.63,0.69}{\textbf{\textit{{#1}}}}}
    
    
    % Define a nice break command that doesn't care if a line doesn't already
    % exist.
    \def\br{\hspace*{\fill} \\* }
    % Math Jax compatibility definitions
    \def\gt{>}
    \def\lt{<}
    \let\Oldtex\TeX
    \let\Oldlatex\LaTeX
    \renewcommand{\TeX}{\textrm{\Oldtex}}
    \renewcommand{\LaTeX}{\textrm{\Oldlatex}}
    % Document parameters
    % Document title
    \title{PairsTrading}
    
    
    
    
    
% Pygments definitions
\makeatletter
\def\PY@reset{\let\PY@it=\relax \let\PY@bf=\relax%
    \let\PY@ul=\relax \let\PY@tc=\relax%
    \let\PY@bc=\relax \let\PY@ff=\relax}
\def\PY@tok#1{\csname PY@tok@#1\endcsname}
\def\PY@toks#1+{\ifx\relax#1\empty\else%
    \PY@tok{#1}\expandafter\PY@toks\fi}
\def\PY@do#1{\PY@bc{\PY@tc{\PY@ul{%
    \PY@it{\PY@bf{\PY@ff{#1}}}}}}}
\def\PY#1#2{\PY@reset\PY@toks#1+\relax+\PY@do{#2}}

\expandafter\def\csname PY@tok@w\endcsname{\def\PY@tc##1{\textcolor[rgb]{0.73,0.73,0.73}{##1}}}
\expandafter\def\csname PY@tok@c\endcsname{\let\PY@it=\textit\def\PY@tc##1{\textcolor[rgb]{0.25,0.50,0.50}{##1}}}
\expandafter\def\csname PY@tok@cp\endcsname{\def\PY@tc##1{\textcolor[rgb]{0.74,0.48,0.00}{##1}}}
\expandafter\def\csname PY@tok@k\endcsname{\let\PY@bf=\textbf\def\PY@tc##1{\textcolor[rgb]{0.00,0.50,0.00}{##1}}}
\expandafter\def\csname PY@tok@kp\endcsname{\def\PY@tc##1{\textcolor[rgb]{0.00,0.50,0.00}{##1}}}
\expandafter\def\csname PY@tok@kt\endcsname{\def\PY@tc##1{\textcolor[rgb]{0.69,0.00,0.25}{##1}}}
\expandafter\def\csname PY@tok@o\endcsname{\def\PY@tc##1{\textcolor[rgb]{0.40,0.40,0.40}{##1}}}
\expandafter\def\csname PY@tok@ow\endcsname{\let\PY@bf=\textbf\def\PY@tc##1{\textcolor[rgb]{0.67,0.13,1.00}{##1}}}
\expandafter\def\csname PY@tok@nb\endcsname{\def\PY@tc##1{\textcolor[rgb]{0.00,0.50,0.00}{##1}}}
\expandafter\def\csname PY@tok@nf\endcsname{\def\PY@tc##1{\textcolor[rgb]{0.00,0.00,1.00}{##1}}}
\expandafter\def\csname PY@tok@nc\endcsname{\let\PY@bf=\textbf\def\PY@tc##1{\textcolor[rgb]{0.00,0.00,1.00}{##1}}}
\expandafter\def\csname PY@tok@nn\endcsname{\let\PY@bf=\textbf\def\PY@tc##1{\textcolor[rgb]{0.00,0.00,1.00}{##1}}}
\expandafter\def\csname PY@tok@ne\endcsname{\let\PY@bf=\textbf\def\PY@tc##1{\textcolor[rgb]{0.82,0.25,0.23}{##1}}}
\expandafter\def\csname PY@tok@nv\endcsname{\def\PY@tc##1{\textcolor[rgb]{0.10,0.09,0.49}{##1}}}
\expandafter\def\csname PY@tok@no\endcsname{\def\PY@tc##1{\textcolor[rgb]{0.53,0.00,0.00}{##1}}}
\expandafter\def\csname PY@tok@nl\endcsname{\def\PY@tc##1{\textcolor[rgb]{0.63,0.63,0.00}{##1}}}
\expandafter\def\csname PY@tok@ni\endcsname{\let\PY@bf=\textbf\def\PY@tc##1{\textcolor[rgb]{0.60,0.60,0.60}{##1}}}
\expandafter\def\csname PY@tok@na\endcsname{\def\PY@tc##1{\textcolor[rgb]{0.49,0.56,0.16}{##1}}}
\expandafter\def\csname PY@tok@nt\endcsname{\let\PY@bf=\textbf\def\PY@tc##1{\textcolor[rgb]{0.00,0.50,0.00}{##1}}}
\expandafter\def\csname PY@tok@nd\endcsname{\def\PY@tc##1{\textcolor[rgb]{0.67,0.13,1.00}{##1}}}
\expandafter\def\csname PY@tok@s\endcsname{\def\PY@tc##1{\textcolor[rgb]{0.73,0.13,0.13}{##1}}}
\expandafter\def\csname PY@tok@sd\endcsname{\let\PY@it=\textit\def\PY@tc##1{\textcolor[rgb]{0.73,0.13,0.13}{##1}}}
\expandafter\def\csname PY@tok@si\endcsname{\let\PY@bf=\textbf\def\PY@tc##1{\textcolor[rgb]{0.73,0.40,0.53}{##1}}}
\expandafter\def\csname PY@tok@se\endcsname{\let\PY@bf=\textbf\def\PY@tc##1{\textcolor[rgb]{0.73,0.40,0.13}{##1}}}
\expandafter\def\csname PY@tok@sr\endcsname{\def\PY@tc##1{\textcolor[rgb]{0.73,0.40,0.53}{##1}}}
\expandafter\def\csname PY@tok@ss\endcsname{\def\PY@tc##1{\textcolor[rgb]{0.10,0.09,0.49}{##1}}}
\expandafter\def\csname PY@tok@sx\endcsname{\def\PY@tc##1{\textcolor[rgb]{0.00,0.50,0.00}{##1}}}
\expandafter\def\csname PY@tok@m\endcsname{\def\PY@tc##1{\textcolor[rgb]{0.40,0.40,0.40}{##1}}}
\expandafter\def\csname PY@tok@gh\endcsname{\let\PY@bf=\textbf\def\PY@tc##1{\textcolor[rgb]{0.00,0.00,0.50}{##1}}}
\expandafter\def\csname PY@tok@gu\endcsname{\let\PY@bf=\textbf\def\PY@tc##1{\textcolor[rgb]{0.50,0.00,0.50}{##1}}}
\expandafter\def\csname PY@tok@gd\endcsname{\def\PY@tc##1{\textcolor[rgb]{0.63,0.00,0.00}{##1}}}
\expandafter\def\csname PY@tok@gi\endcsname{\def\PY@tc##1{\textcolor[rgb]{0.00,0.63,0.00}{##1}}}
\expandafter\def\csname PY@tok@gr\endcsname{\def\PY@tc##1{\textcolor[rgb]{1.00,0.00,0.00}{##1}}}
\expandafter\def\csname PY@tok@ge\endcsname{\let\PY@it=\textit}
\expandafter\def\csname PY@tok@gs\endcsname{\let\PY@bf=\textbf}
\expandafter\def\csname PY@tok@gp\endcsname{\let\PY@bf=\textbf\def\PY@tc##1{\textcolor[rgb]{0.00,0.00,0.50}{##1}}}
\expandafter\def\csname PY@tok@go\endcsname{\def\PY@tc##1{\textcolor[rgb]{0.53,0.53,0.53}{##1}}}
\expandafter\def\csname PY@tok@gt\endcsname{\def\PY@tc##1{\textcolor[rgb]{0.00,0.27,0.87}{##1}}}
\expandafter\def\csname PY@tok@err\endcsname{\def\PY@bc##1{\setlength{\fboxsep}{0pt}\fcolorbox[rgb]{1.00,0.00,0.00}{1,1,1}{\strut ##1}}}
\expandafter\def\csname PY@tok@kc\endcsname{\let\PY@bf=\textbf\def\PY@tc##1{\textcolor[rgb]{0.00,0.50,0.00}{##1}}}
\expandafter\def\csname PY@tok@kd\endcsname{\let\PY@bf=\textbf\def\PY@tc##1{\textcolor[rgb]{0.00,0.50,0.00}{##1}}}
\expandafter\def\csname PY@tok@kn\endcsname{\let\PY@bf=\textbf\def\PY@tc##1{\textcolor[rgb]{0.00,0.50,0.00}{##1}}}
\expandafter\def\csname PY@tok@kr\endcsname{\let\PY@bf=\textbf\def\PY@tc##1{\textcolor[rgb]{0.00,0.50,0.00}{##1}}}
\expandafter\def\csname PY@tok@bp\endcsname{\def\PY@tc##1{\textcolor[rgb]{0.00,0.50,0.00}{##1}}}
\expandafter\def\csname PY@tok@fm\endcsname{\def\PY@tc##1{\textcolor[rgb]{0.00,0.00,1.00}{##1}}}
\expandafter\def\csname PY@tok@vc\endcsname{\def\PY@tc##1{\textcolor[rgb]{0.10,0.09,0.49}{##1}}}
\expandafter\def\csname PY@tok@vg\endcsname{\def\PY@tc##1{\textcolor[rgb]{0.10,0.09,0.49}{##1}}}
\expandafter\def\csname PY@tok@vi\endcsname{\def\PY@tc##1{\textcolor[rgb]{0.10,0.09,0.49}{##1}}}
\expandafter\def\csname PY@tok@vm\endcsname{\def\PY@tc##1{\textcolor[rgb]{0.10,0.09,0.49}{##1}}}
\expandafter\def\csname PY@tok@sa\endcsname{\def\PY@tc##1{\textcolor[rgb]{0.73,0.13,0.13}{##1}}}
\expandafter\def\csname PY@tok@sb\endcsname{\def\PY@tc##1{\textcolor[rgb]{0.73,0.13,0.13}{##1}}}
\expandafter\def\csname PY@tok@sc\endcsname{\def\PY@tc##1{\textcolor[rgb]{0.73,0.13,0.13}{##1}}}
\expandafter\def\csname PY@tok@dl\endcsname{\def\PY@tc##1{\textcolor[rgb]{0.73,0.13,0.13}{##1}}}
\expandafter\def\csname PY@tok@s2\endcsname{\def\PY@tc##1{\textcolor[rgb]{0.73,0.13,0.13}{##1}}}
\expandafter\def\csname PY@tok@sh\endcsname{\def\PY@tc##1{\textcolor[rgb]{0.73,0.13,0.13}{##1}}}
\expandafter\def\csname PY@tok@s1\endcsname{\def\PY@tc##1{\textcolor[rgb]{0.73,0.13,0.13}{##1}}}
\expandafter\def\csname PY@tok@mb\endcsname{\def\PY@tc##1{\textcolor[rgb]{0.40,0.40,0.40}{##1}}}
\expandafter\def\csname PY@tok@mf\endcsname{\def\PY@tc##1{\textcolor[rgb]{0.40,0.40,0.40}{##1}}}
\expandafter\def\csname PY@tok@mh\endcsname{\def\PY@tc##1{\textcolor[rgb]{0.40,0.40,0.40}{##1}}}
\expandafter\def\csname PY@tok@mi\endcsname{\def\PY@tc##1{\textcolor[rgb]{0.40,0.40,0.40}{##1}}}
\expandafter\def\csname PY@tok@il\endcsname{\def\PY@tc##1{\textcolor[rgb]{0.40,0.40,0.40}{##1}}}
\expandafter\def\csname PY@tok@mo\endcsname{\def\PY@tc##1{\textcolor[rgb]{0.40,0.40,0.40}{##1}}}
\expandafter\def\csname PY@tok@ch\endcsname{\let\PY@it=\textit\def\PY@tc##1{\textcolor[rgb]{0.25,0.50,0.50}{##1}}}
\expandafter\def\csname PY@tok@cm\endcsname{\let\PY@it=\textit\def\PY@tc##1{\textcolor[rgb]{0.25,0.50,0.50}{##1}}}
\expandafter\def\csname PY@tok@cpf\endcsname{\let\PY@it=\textit\def\PY@tc##1{\textcolor[rgb]{0.25,0.50,0.50}{##1}}}
\expandafter\def\csname PY@tok@c1\endcsname{\let\PY@it=\textit\def\PY@tc##1{\textcolor[rgb]{0.25,0.50,0.50}{##1}}}
\expandafter\def\csname PY@tok@cs\endcsname{\let\PY@it=\textit\def\PY@tc##1{\textcolor[rgb]{0.25,0.50,0.50}{##1}}}

\def\PYZbs{\char`\\}
\def\PYZus{\char`\_}
\def\PYZob{\char`\{}
\def\PYZcb{\char`\}}
\def\PYZca{\char`\^}
\def\PYZam{\char`\&}
\def\PYZlt{\char`\<}
\def\PYZgt{\char`\>}
\def\PYZsh{\char`\#}
\def\PYZpc{\char`\%}
\def\PYZdl{\char`\$}
\def\PYZhy{\char`\-}
\def\PYZsq{\char`\'}
\def\PYZdq{\char`\"}
\def\PYZti{\char`\~}
% for compatibility with earlier versions
\def\PYZat{@}
\def\PYZlb{[}
\def\PYZrb{]}
\makeatother


    % For linebreaks inside Verbatim environment from package fancyvrb. 
    \makeatletter
        \newbox\Wrappedcontinuationbox 
        \newbox\Wrappedvisiblespacebox 
        \newcommand*\Wrappedvisiblespace {\textcolor{red}{\textvisiblespace}} 
        \newcommand*\Wrappedcontinuationsymbol {\textcolor{red}{\llap{\tiny$\m@th\hookrightarrow$}}} 
        \newcommand*\Wrappedcontinuationindent {3ex } 
        \newcommand*\Wrappedafterbreak {\kern\Wrappedcontinuationindent\copy\Wrappedcontinuationbox} 
        % Take advantage of the already applied Pygments mark-up to insert 
        % potential linebreaks for TeX processing. 
        %        {, <, #, %, $, ' and ": go to next line. 
        %        _, }, ^, &, >, - and ~: stay at end of broken line. 
        % Use of \textquotesingle for straight quote. 
        \newcommand*\Wrappedbreaksatspecials {% 
            \def\PYGZus{\discretionary{\char`\_}{\Wrappedafterbreak}{\char`\_}}% 
            \def\PYGZob{\discretionary{}{\Wrappedafterbreak\char`\{}{\char`\{}}% 
            \def\PYGZcb{\discretionary{\char`\}}{\Wrappedafterbreak}{\char`\}}}% 
            \def\PYGZca{\discretionary{\char`\^}{\Wrappedafterbreak}{\char`\^}}% 
            \def\PYGZam{\discretionary{\char`\&}{\Wrappedafterbreak}{\char`\&}}% 
            \def\PYGZlt{\discretionary{}{\Wrappedafterbreak\char`\<}{\char`\<}}% 
            \def\PYGZgt{\discretionary{\char`\>}{\Wrappedafterbreak}{\char`\>}}% 
            \def\PYGZsh{\discretionary{}{\Wrappedafterbreak\char`\#}{\char`\#}}% 
            \def\PYGZpc{\discretionary{}{\Wrappedafterbreak\char`\%}{\char`\%}}% 
            \def\PYGZdl{\discretionary{}{\Wrappedafterbreak\char`\$}{\char`\$}}% 
            \def\PYGZhy{\discretionary{\char`\-}{\Wrappedafterbreak}{\char`\-}}% 
            \def\PYGZsq{\discretionary{}{\Wrappedafterbreak\textquotesingle}{\textquotesingle}}% 
            \def\PYGZdq{\discretionary{}{\Wrappedafterbreak\char`\"}{\char`\"}}% 
            \def\PYGZti{\discretionary{\char`\~}{\Wrappedafterbreak}{\char`\~}}% 
        } 
        % Some characters . , ; ? ! / are not pygmentized. 
        % This macro makes them "active" and they will insert potential linebreaks 
        \newcommand*\Wrappedbreaksatpunct {% 
            \lccode`\~`\.\lowercase{\def~}{\discretionary{\hbox{\char`\.}}{\Wrappedafterbreak}{\hbox{\char`\.}}}% 
            \lccode`\~`\,\lowercase{\def~}{\discretionary{\hbox{\char`\,}}{\Wrappedafterbreak}{\hbox{\char`\,}}}% 
            \lccode`\~`\;\lowercase{\def~}{\discretionary{\hbox{\char`\;}}{\Wrappedafterbreak}{\hbox{\char`\;}}}% 
            \lccode`\~`\:\lowercase{\def~}{\discretionary{\hbox{\char`\:}}{\Wrappedafterbreak}{\hbox{\char`\:}}}% 
            \lccode`\~`\?\lowercase{\def~}{\discretionary{\hbox{\char`\?}}{\Wrappedafterbreak}{\hbox{\char`\?}}}% 
            \lccode`\~`\!\lowercase{\def~}{\discretionary{\hbox{\char`\!}}{\Wrappedafterbreak}{\hbox{\char`\!}}}% 
            \lccode`\~`\/\lowercase{\def~}{\discretionary{\hbox{\char`\/}}{\Wrappedafterbreak}{\hbox{\char`\/}}}% 
            \catcode`\.\active
            \catcode`\,\active 
            \catcode`\;\active
            \catcode`\:\active
            \catcode`\?\active
            \catcode`\!\active
            \catcode`\/\active 
            \lccode`\~`\~ 	
        }
    \makeatother

    \let\OriginalVerbatim=\Verbatim
    \makeatletter
    \renewcommand{\Verbatim}[1][1]{%
        %\parskip\z@skip
        \sbox\Wrappedcontinuationbox {\Wrappedcontinuationsymbol}%
        \sbox\Wrappedvisiblespacebox {\FV@SetupFont\Wrappedvisiblespace}%
        \def\FancyVerbFormatLine ##1{\hsize\linewidth
            \vtop{\raggedright\hyphenpenalty\z@\exhyphenpenalty\z@
                \doublehyphendemerits\z@\finalhyphendemerits\z@
                \strut ##1\strut}%
        }%
        % If the linebreak is at a space, the latter will be displayed as visible
        % space at end of first line, and a continuation symbol starts next line.
        % Stretch/shrink are however usually zero for typewriter font.
        \def\FV@Space {%
            \nobreak\hskip\z@ plus\fontdimen3\font minus\fontdimen4\font
            \discretionary{\copy\Wrappedvisiblespacebox}{\Wrappedafterbreak}
            {\kern\fontdimen2\font}%
        }%
        
        % Allow breaks at special characters using \PYG... macros.
        \Wrappedbreaksatspecials
        % Breaks at punctuation characters . , ; ? ! and / need catcode=\active 	
        \OriginalVerbatim[#1,codes*=\Wrappedbreaksatpunct]%
    }
    \makeatother

    % Exact colors from NB
    \definecolor{incolor}{HTML}{303F9F}
    \definecolor{outcolor}{HTML}{D84315}
    \definecolor{cellborder}{HTML}{CFCFCF}
    \definecolor{cellbackground}{HTML}{F7F7F7}
    
    % prompt
    \makeatletter
    \newcommand{\boxspacing}{\kern\kvtcb@left@rule\kern\kvtcb@boxsep}
    \makeatother
    \newcommand{\prompt}[4]{
        \ttfamily\llap{{\color{#2}[#3]:\hspace{3pt}#4}}\vspace{-\baselineskip}
    }
    

    
    % Prevent overflowing lines due to hard-to-break entities
    \sloppy 
    % Setup hyperref package
    \hypersetup{
      breaklinks=true,  % so long urls are correctly broken across lines
      colorlinks=true,
      urlcolor=urlcolor,
      linkcolor=linkcolor,
      citecolor=citecolor,
      }
    % Slightly bigger margins than the latex defaults
    
    \geometry{verbose,tmargin=1in,bmargin=1in,lmargin=1in,rmargin=1in}
    
    

\begin{document}
    
    \maketitle
    
    

    
    \hypertarget{pairs-trading}{%
\section{Pairs Trading}\label{pairs-trading}}

This trading strategy is based upon mathematical analysis, it states
that is you have a pair of assets \(X\) and \(Y\) that have some
undelying economic link, for example two companies that manufacture the
same product, said economic link can be represented with a methematical
model upon which we can make trades.

\hypertarget{cointegration}{%
\subsection{Cointegration}\label{cointegration}}

Cointegration is the fundamental way in which we will find these pairs
of assets, therefore we must exlain what cointegration is.

We will start by defining a stationary process, we say something is
stationary in the wide sense when it's joint probability distribution
does not change when shifted in time, meaning that variables such as
mean and variance will be constant over time, this is relevant, because
most statistical tests assume stationarity, and therefore not having
stationary data could lead to results that are not useful, or even worse
could result in errors. An example of this is the use of the mean of a
series as a prediction for future behaviour; in a stationary series the
means does an okay job at predictiong future values, not great but
useful, while the mean of a non-stationary function tells us nothing
about its future behaviour. In order to test for stationarity we use
\(p-values\) acording to the augmented Dickey-Fuller test. In gneral a
\(p-value\) of less than 5\% means the null hypothesis can be rejected
(the null hypothesis being that the data are non-stationary).

There is another important definition that must be made before we go on
to explain what cointegrations is, and that would be what the orders of
integration are. The fist order of integration is \(I(0)\), this order
of integration is really hard to test for given any slightly large
dataset, but any stationary process has an integration order of 0. In
order to get a series with an order of integration of 1, what must be
done is take an \(I(0)\) series and use discrete integration on it
(which would be to commulatively sum it). And so on, in order to get an
\(I(2)\) series we use discrete integration on a \(I(1)\) series, and to
get an \(I(3)\) series we commulatively sun an \(I(2)\) series.

Now, finally, for some set of time series \((X_1, X_2, ..., X_k)\) if
all the time series are \(I(1)\) and there is a linear combination of
them that is \(I(0)\), then we can say that this set of time series is
cointegrated. Intitively we can state that there is cointegration
between two series if their difference is mean reverting, which means
that it is centered around and hovers around the mean of the series
abtaining by substracting the two cointegrated series. For more specific
purposes, we will use the Engle-Granger two-step cointegration test
foind in the \emph{statsmodels} python library. This is the base of
pairs trading, because if we find two assets with cointegrated
price-time series, we have reason to believe that there is a relyable
economic link betwen these two assets.

    \begin{tcolorbox}[breakable, size=fbox, boxrule=1pt, pad at break*=1mm,colback=cellbackground, colframe=cellborder]
\prompt{In}{incolor}{1}{\boxspacing}
\begin{Verbatim}[commandchars=\\\{\}]
\PY{k+kn}{import} \PY{n+nn}{numpy} \PY{k}{as} \PY{n+nn}{np}
\PY{k+kn}{import} \PY{n+nn}{pandas} \PY{k}{as} \PY{n+nn}{pd}
\PY{k+kn}{import} \PY{n+nn}{matplotlib}\PY{n+nn}{.}\PY{n+nn}{pyplot} \PY{k}{as} \PY{n+nn}{plt}
\PY{k+kn}{import} \PY{n+nn}{yfinance} \PY{k}{as} \PY{n+nn}{yf}
\PY{k+kn}{import} \PY{n+nn}{seaborn}
\PY{k+kn}{import} \PY{n+nn}{statsmodels}
\PY{k+kn}{import} \PY{n+nn}{statsmodels}\PY{n+nn}{.}\PY{n+nn}{api} \PY{k}{as} \PY{n+nn}{sm}
\PY{k+kn}{from} \PY{n+nn}{statsmodels}\PY{n+nn}{.}\PY{n+nn}{tsa}\PY{n+nn}{.}\PY{n+nn}{stattools} \PY{k+kn}{import} \PY{n}{coint}
\end{Verbatim}
\end{tcolorbox}

    \hypertarget{finding-assets-with-cointegrated-prices}{%
\section{Finding Assets with Cointegrated
Prices}\label{finding-assets-with-cointegrated-prices}}

In order to do this we will takje a two step approach. In order to avoid
the multiple comparison bias, we will begin by defining a group of
assets that we have reason to believe have cointegrated prices, this
reason could be either that they both sell or produce the same product,
or that they both belong to the same supply chain. Then, after
identifying this set of assets we will apply the Engle-Gartener two-step
cointegration method to every possible pair.

For this example we will use assets that belong to the oit industry. In
this case: Occidental Petroleum Corporation (OXY), Oasis Petroleum
Inc.~(OAS), Callon Petroleum Company (CPE), Abraxas Petroleum
Corporation (AXAS), Marathon Petroleum Corporation (MPC).

    \begin{tcolorbox}[breakable, size=fbox, boxrule=1pt, pad at break*=1mm,colback=cellbackground, colframe=cellborder]
\prompt{In}{incolor}{46}{\boxspacing}
\begin{Verbatim}[commandchars=\\\{\}]
\PY{c+c1}{\PYZsh{} Lets start by finding stock with cointegrated proces}

\PY{n}{possible\PYZus{}pairs} \PY{o}{=} \PY{p}{[}\PY{l+s+s1}{\PYZsq{}}\PY{l+s+s1}{OXY}\PY{l+s+s1}{\PYZsq{}}\PY{p}{,} \PY{l+s+s1}{\PYZsq{}}\PY{l+s+s1}{OAS}\PY{l+s+s1}{\PYZsq{}}\PY{p}{,} \PY{l+s+s1}{\PYZsq{}}\PY{l+s+s1}{CPE}\PY{l+s+s1}{\PYZsq{}}\PY{p}{,} \PY{l+s+s1}{\PYZsq{}}\PY{l+s+s1}{AXAS}\PY{l+s+s1}{\PYZsq{}}\PY{p}{,} \PY{l+s+s1}{\PYZsq{}}\PY{l+s+s1}{MPC}\PY{l+s+s1}{\PYZsq{}}\PY{p}{]} \PY{c+c1}{\PYZsh{} Make a list with all the relevant assets}
\PY{n}{financial\PYZus{}information} \PY{o}{=} \PY{n}{yf}\PY{o}{.}\PY{n}{download}\PY{p}{(}\PY{n}{possible\PYZus{}pairs}\PY{p}{,} \PY{n}{start} \PY{o}{=} \PY{l+s+s1}{\PYZsq{}}\PY{l+s+s1}{2018\PYZhy{}07\PYZhy{}15}\PY{l+s+s1}{\PYZsq{}}\PY{p}{,} \PY{n}{end} \PY{o}{=} \PY{l+s+s1}{\PYZsq{}}\PY{l+s+s1}{2019\PYZhy{}07\PYZhy{}14}\PY{l+s+s1}{\PYZsq{}}\PY{p}{)} \PY{c+c1}{\PYZsh{} Retrieve the financial information of the stock from yahoo finance}
\PY{n}{financial\PYZus{}information} \PY{o}{=} \PY{n}{financial\PYZus{}information}\PY{p}{[}\PY{l+s+s1}{\PYZsq{}}\PY{l+s+s1}{Close}\PY{l+s+s1}{\PYZsq{}}\PY{p}{]} \PY{c+c1}{\PYZsh{} Drop every column that is not closing price}

\PY{n}{financial\PYZus{}information}\PY{o}{.}\PY{n}{head}\PY{p}{(}\PY{p}{)}
\end{Verbatim}
\end{tcolorbox}

    \begin{Verbatim}[commandchars=\\\{\}]
[*********************100\%***********************]  5 of 5 completed
    \end{Verbatim}

            \begin{tcolorbox}[breakable, size=fbox, boxrule=.5pt, pad at break*=1mm, opacityfill=0]
\prompt{Out}{outcolor}{46}{\boxspacing}
\begin{Verbatim}[commandchars=\\\{\}]
            AXAS         CPE        MPC    OAS        OXY
Date
2018-07-16  2.77  108.599998  71.419998  11.90  82.529999
2018-07-17  2.84  108.699997  71.699997  11.89  82.690002
2018-07-18  2.85  110.099998  72.559998  11.90  82.959999
2018-07-19  2.87  112.500000  72.860001  12.08  82.980003
2018-07-20  2.76  111.099998  72.519997  12.08  83.080002
\end{Verbatim}
\end{tcolorbox}
        
    \begin{tcolorbox}[breakable, size=fbox, boxrule=1pt, pad at break*=1mm,colback=cellbackground, colframe=cellborder]
\prompt{In}{incolor}{51}{\boxspacing}
\begin{Verbatim}[commandchars=\\\{\}]
\PY{k}{def} \PY{n+nf}{test\PYZus{}cointegration}\PY{p}{(}\PY{n}{financial\PYZus{}information}\PY{p}{)}\PY{p}{:}
    \PY{n}{number\PYZus{}of\PYZus{}assets} \PY{o}{=} \PY{n}{financial\PYZus{}information}\PY{o}{.}\PY{n}{shape}\PY{p}{[}\PY{l+m+mi}{1}\PY{p}{]}
    \PY{n}{score\PYZus{}matrix} \PY{o}{=} \PY{n}{np}\PY{o}{.}\PY{n}{zeros}\PY{p}{(}\PY{p}{(}\PY{n}{number\PYZus{}of\PYZus{}assets}\PY{p}{,} \PY{n}{number\PYZus{}of\PYZus{}assets}\PY{p}{)}\PY{p}{)}
    \PY{n}{p\PYZus{}value\PYZus{}matrix} \PY{o}{=} \PY{n}{np}\PY{o}{.}\PY{n}{ones}\PY{p}{(}\PY{p}{(}\PY{n}{number\PYZus{}of\PYZus{}assets}\PY{p}{,} \PY{n}{number\PYZus{}of\PYZus{}assets}\PY{p}{)}\PY{p}{)}
    \PY{n}{keys} \PY{o}{=} \PY{n}{financial\PYZus{}information}\PY{o}{.}\PY{n}{keys}\PY{p}{(}\PY{p}{)}
    \PY{n}{pairs} \PY{o}{=} \PY{p}{[}\PY{p}{]}
    \PY{k}{for} \PY{n}{asset} \PY{o+ow}{in} \PY{n+nb}{range}\PY{p}{(}\PY{n}{number\PYZus{}of\PYZus{}assets}\PY{p}{)}\PY{p}{:}
        \PY{k}{for} \PY{n}{possible\PYZus{}cointegrated\PYZus{}asset} \PY{o+ow}{in} \PY{n+nb}{range}\PY{p}{(}\PY{n}{asset} \PY{o}{+} \PY{l+m+mi}{1}\PY{p}{,} \PY{n}{number\PYZus{}of\PYZus{}assets}\PY{p}{)}\PY{p}{:}
            \PY{n}{asset\PYZus{}1} \PY{o}{=} \PY{n}{financial\PYZus{}information}\PY{p}{[}\PY{n}{keys}\PY{p}{[}\PY{n}{asset}\PY{p}{]}\PY{p}{]}
            \PY{n}{asset\PYZus{}2} \PY{o}{=} \PY{n}{financial\PYZus{}information}\PY{p}{[}\PY{n}{keys}\PY{p}{[}\PY{n}{possible\PYZus{}cointegrated\PYZus{}asset}\PY{p}{]}\PY{p}{]}
            \PY{n}{cointegration\PYZus{}test\PYZus{}result} \PY{o}{=} \PY{n}{coint}\PY{p}{(}\PY{n}{asset\PYZus{}1}\PY{p}{,} \PY{n}{asset\PYZus{}2}\PY{p}{)}
            \PY{n}{score} \PY{o}{=} \PY{n}{cointegration\PYZus{}test\PYZus{}result}\PY{p}{[}\PY{l+m+mi}{0}\PY{p}{]}
            \PY{n}{p\PYZus{}value} \PY{o}{=} \PY{n}{cointegration\PYZus{}test\PYZus{}result}\PY{p}{[}\PY{l+m+mi}{1}\PY{p}{]}
            \PY{n}{score\PYZus{}matrix}\PY{p}{[}\PY{n}{asset}\PY{p}{,} \PY{n}{possible\PYZus{}cointegrated\PYZus{}asset}\PY{p}{]} \PY{o}{=} \PY{n}{score}
            \PY{n}{p\PYZus{}value\PYZus{}matrix}\PY{p}{[}\PY{n}{asset}\PY{p}{,} \PY{n}{possible\PYZus{}cointegrated\PYZus{}asset}\PY{p}{]} \PY{o}{=} \PY{n}{p\PYZus{}value}
            \PY{k}{if} \PY{n}{p\PYZus{}value} \PY{o}{\PYZlt{}} \PY{l+m+mf}{0.01}\PY{p}{:}
                \PY{n}{pairs}\PY{o}{.}\PY{n}{append}\PY{p}{(}\PY{p}{(}\PY{n}{keys}\PY{p}{[}\PY{n}{asset}\PY{p}{]}\PY{p}{,} \PY{n}{keys}\PY{p}{[}\PY{n}{possible\PYZus{}cointegrated\PYZus{}asset}\PY{p}{]}\PY{p}{)}\PY{p}{)}
        \PY{k}{return} \PY{n}{score\PYZus{}matrix}\PY{p}{,} \PY{n}{p\PYZus{}value\PYZus{}matrix}\PY{p}{,} \PY{n}{pairs}
    
\PY{n}{scores}\PY{p}{,} \PY{n}{p\PYZus{}values}\PY{p}{,} \PY{n}{pairs} \PY{o}{=} \PY{n}{test\PYZus{}cointegration}\PY{p}{(}\PY{n}{financial\PYZus{}information}\PY{p}{)}
\PY{n}{seaborn}\PY{o}{.}\PY{n}{heatmap}\PY{p}{(}\PY{n}{p\PYZus{}values}\PY{p}{,} \PY{n}{xticklabels} \PY{o}{=} \PY{n}{financial\PYZus{}information}\PY{o}{.}\PY{n}{columns}\PY{p}{,} \PY{n}{yticklabels} \PY{o}{=} \PY{n}{financial\PYZus{}information}\PY{o}{.}\PY{n}{columns}\PY{p}{,} \PY{n}{cmap} \PY{o}{=} \PY{l+s+s1}{\PYZsq{}}\PY{l+s+s1}{mako}\PY{l+s+s1}{\PYZsq{}}\PY{p}{,} \PY{n}{mask} \PY{o}{=} \PY{p}{(}\PY{n}{p\PYZus{}values} \PY{o}{\PYZgt{}}\PY{o}{=} \PY{l+m+mf}{0.01}\PY{p}{)}\PY{p}{)}
\end{Verbatim}
\end{tcolorbox}

            \begin{tcolorbox}[breakable, size=fbox, boxrule=.5pt, pad at break*=1mm, opacityfill=0]
\prompt{Out}{outcolor}{51}{\boxspacing}
\begin{Verbatim}[commandchars=\\\{\}]
<AxesSubplot:>
\end{Verbatim}
\end{tcolorbox}
        
    \begin{center}
    \adjustimage{max size={0.9\linewidth}{0.9\paperheight}}{output_4_1.png}
    \end{center}
    { \hspace*{\fill} \\}
    
    \hypertarget{trading-strategy}{%
\section{Trading Strategy}\label{trading-strategy}}

In our corrintegration analysis we have found that AXAS and MPC, and
AXAS and OXY are possible pairs with cointegrated priceses, which means
that we can use either MPC or OXY in order to calculate a predicted
trend for AXAS, for the following example we will use MPC. Knowing this
we can go ahead and use our knowledge of cointegration to device a price
prediction strategy. For this one in partícular we will use \(z-values\)
which tell us how many standar deviantiosns away from the mean a value
is. This values are calculated as follows.

\(Z = \frac{x - \mu}{\sigma}\)

Where:

\$ Z = \$ standard score

\$ x = \$ observed value

\$ \mu = \$ mean of the sample

\$ \sigma = \$ standard deviantion of the sample

This \(z-values\) will help us assess the endency of the stock in the
following way, if the \(z-value > 1\) the asset's price will fall, but
if \(z-value < -1\) the asset's price will rise. This is due to the mean
reverting secuence we will create by substracting MPC from AXAS, if the
observed value is far below the mean, we know it will have to rise in
order to keep up with its pair, and the opposite willl happen if the
observed value is considerably higher than the mean, it will have to
fall in order to revert to the average.

    \begin{tcolorbox}[breakable, size=fbox, boxrule=1pt, pad at break*=1mm,colback=cellbackground, colframe=cellborder]
\prompt{In}{incolor}{79}{\boxspacing}
\begin{Verbatim}[commandchars=\\\{\}]
\PY{n}{normalization} \PY{o}{=} \PY{n}{financial\PYZus{}information}\PY{p}{[}\PY{l+s+s1}{\PYZsq{}}\PY{l+s+s1}{AXAS}\PY{l+s+s1}{\PYZsq{}}\PY{p}{]} \PY{o}{\PYZhy{}} \PY{n}{financial\PYZus{}information}\PY{p}{[}\PY{l+s+s1}{\PYZsq{}}\PY{l+s+s1}{MPC}\PY{l+s+s1}{\PYZsq{}}\PY{p}{]} 

\PY{n}{normalization\PYZus{}moving\PYZus{}average\PYZus{}1} \PY{o}{=} \PY{n}{normalization}\PY{o}{.}\PY{n}{rolling}\PY{p}{(}\PY{l+m+mi}{1}\PY{p}{)}\PY{o}{.}\PY{n}{mean}\PY{p}{(}\PY{p}{)}
\PY{n}{normalization\PYZus{}moving\PYZus{}average\PYZus{}30} \PY{o}{=} \PY{n}{normalization}\PY{o}{.}\PY{n}{rolling}\PY{p}{(}\PY{l+m+mi}{30}\PY{p}{)}\PY{o}{.}\PY{n}{mean}\PY{p}{(}\PY{p}{)}
\PY{n}{normalization\PYZus{}standar\PYZus{}deviation\PYZus{}30} \PY{o}{=} \PY{n}{normalization}\PY{o}{.}\PY{n}{rolling}\PY{p}{(}\PY{l+m+mi}{30}\PY{p}{)}\PY{o}{.}\PY{n}{std}\PY{p}{(}\PY{p}{)}

\PY{n}{z\PYZus{}values} \PY{o}{=} \PY{p}{(}\PY{n}{normalization\PYZus{}moving\PYZus{}average\PYZus{}1}\PY{o}{\PYZhy{}}\PY{n}{normalization\PYZus{}moving\PYZus{}average\PYZus{}30}\PY{p}{)}\PY{o}{/}\PY{n}{normalization\PYZus{}standar\PYZus{}deviation\PYZus{}30}

\PY{n}{z\PYZus{}values}\PY{o}{.}\PY{n}{plot}\PY{p}{(}\PY{n}{label} \PY{o}{=} \PY{l+s+s1}{\PYZsq{}}\PY{l+s+s1}{Rolling 20 day z score}\PY{l+s+s1}{\PYZsq{}}\PY{p}{)}
\PY{n}{financial\PYZus{}information}\PY{p}{[}\PY{l+s+s1}{\PYZsq{}}\PY{l+s+s1}{AXAS}\PY{l+s+s1}{\PYZsq{}}\PY{p}{]}\PY{o}{.}\PY{n}{plot}\PY{p}{(}\PY{n}{label} \PY{o}{=} \PY{l+s+s1}{\PYZsq{}}\PY{l+s+s1}{Stock 2}\PY{l+s+s1}{\PYZsq{}}\PY{p}{)}
\PY{n}{plt}\PY{o}{.}\PY{n}{axhline}\PY{p}{(}\PY{n}{z\PYZus{}values}\PY{o}{.}\PY{n}{mean}\PY{p}{(}\PY{p}{)}\PY{p}{,} \PY{n}{color} \PY{o}{=} \PY{l+s+s1}{\PYZsq{}}\PY{l+s+s1}{b}\PY{l+s+s1}{\PYZsq{}}\PY{p}{)}
\PY{n}{plt}\PY{o}{.}\PY{n}{axhline}\PY{p}{(}\PY{l+m+mf}{1.0}\PY{p}{,} \PY{n}{color} \PY{o}{=} \PY{l+s+s1}{\PYZsq{}}\PY{l+s+s1}{r}\PY{l+s+s1}{\PYZsq{}}\PY{p}{,} \PY{n}{ls} \PY{o}{=} \PY{l+s+s1}{\PYZsq{}}\PY{l+s+s1}{\PYZhy{}\PYZhy{}}\PY{l+s+s1}{\PYZsq{}}\PY{p}{)}
\PY{n}{plt}\PY{o}{.}\PY{n}{axhline}\PY{p}{(}\PY{o}{\PYZhy{}}\PY{l+m+mf}{1.0}\PY{p}{,} \PY{n}{color} \PY{o}{=} \PY{l+s+s1}{\PYZsq{}}\PY{l+s+s1}{r}\PY{l+s+s1}{\PYZsq{}}\PY{p}{,} \PY{n}{ls} \PY{o}{=} \PY{l+s+s1}{\PYZsq{}}\PY{l+s+s1}{\PYZhy{}\PYZhy{}}\PY{l+s+s1}{\PYZsq{}}\PY{p}{)}

\PY{n}{financial\PYZus{}information}\PY{p}{[}\PY{l+s+s1}{\PYZsq{}}\PY{l+s+s1}{AXAS}\PY{l+s+s1}{\PYZsq{}}\PY{p}{]}\PY{o}{.}\PY{n}{plot}\PY{p}{(}\PY{n}{label} \PY{o}{=} \PY{l+s+s1}{\PYZsq{}}\PY{l+s+s1}{Stock 2}\PY{l+s+s1}{\PYZsq{}}\PY{p}{)}
\end{Verbatim}
\end{tcolorbox}

            \begin{tcolorbox}[breakable, size=fbox, boxrule=.5pt, pad at break*=1mm, opacityfill=0]
\prompt{Out}{outcolor}{79}{\boxspacing}
\begin{Verbatim}[commandchars=\\\{\}]
<AxesSubplot:xlabel='Date'>
\end{Verbatim}
\end{tcolorbox}
        
    \begin{center}
    \adjustimage{max size={0.9\linewidth}{0.9\paperheight}}{output_6_1.png}
    \end{center}
    { \hspace*{\fill} \\}
    
    \hypertarget{limitation}{%
\section{Limitation}\label{limitation}}

By doing this, we have learned that pairs trading is a rather good
method for price prediction, but it has a very disticnt limitation,
there is no haw of accurately predicting price behaviour long-term. This
is due to our inhability to predict the \(z-values\) in the long-term,
in orther to do this we would need to incorporate machine learning
elemts to this algorithm. For the moment, we can conclude that
pair-trading is a very useful way of predicting short-term rises and
falls.

    \begin{tcolorbox}[breakable, size=fbox, boxrule=1pt, pad at break*=1mm,colback=cellbackground, colframe=cellborder]
\prompt{In}{incolor}{ }{\boxspacing}
\begin{Verbatim}[commandchars=\\\{\}]

\end{Verbatim}
\end{tcolorbox}


    % Add a bibliography block to the postdoc
    
    
    
\end{document}
